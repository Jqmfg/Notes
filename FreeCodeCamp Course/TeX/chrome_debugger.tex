\documentclass{article}

\title{Chrome Debugger}
\date{4-14-2015}
\author{David Thien}

\begin{document}
	\pagenumbering{gobble}
	\maketitle
	\newpage
	\tableofcontents
	\newpage
	\pagenumbering{arabic}

	\section{Elements Tab}
	\subsection{Basics}
	To open developer tools we will click on settings->tools->developer tools or use the shortcut ctr+shift+I or f12.	The elements panel displays a representation of the DOM(Document Object Model). You can also right click on a specific element and click inspect element to highlight the html and show the css. You can also select the element select button (magnifying glass) to select an element. You can change elements by selecting the element and then editing the DOM. This is done in the elements panel. Here, you can also add or edit attributes. To add or edit attributes, right click on an element and then choose add or edit element. From here, you can also choose to edit as html. You can even click and drag elements to change their position on the page. Pressing delete will delete the element.
	\subsection{CSS Changes}
	In addition to editing the DOM, you can also make changes to the CSS. The styles tab displays the computed styles from the element and will show all of the styles from all classes. From here, you can:
	\begin{itemize}
		\item  Display CSS rules for active element
		\item  Enable/disable properties
		\item  Edit rules for pseudo-classes
		\item  Link directly to css source
	\end{itemize}
	The most specific rules will be displayed first. The check-boxes next to they styles can disable styles as well. You can double click on the value to change it, or  delete the values. To add a style, click on the + at the top right. Here, you can add styles with autocomplete support. Using the button the right of the + icon in the top right corner, you can  change the settings for where the mouse is. Possible styles are
  	\begin{itemize}
		\item active
  		\item focus
  		\item hover
  		\item visited
	\end{itemize}
You can click on the color swatch to the left of colors to get a color picker. The style view also lets you know where a specific style came form. There will be a file link on the right side that will take you to that file that brings up the sources panel.
	\section{Sources Tab}
	\subsection{Basics}
	The sources tab lets you
	\begin{itemize}
		\item modify application source files
		\item export changes
		\item track file versions
	\end{itemize}
Clicking a file name in the style section of the elements panel will open that file in the sources panel. The page will instantly change when you make changes. There will be a star that is over the .css from unsaved changes. Note that this only saves to the browser. You can right click and see local modifications that have been made including a revision history. There will also be a revision history to toggle changes on and off. Additionally, there is a revert button that will put the file in its original state. You can then save back to the original source file from dev tools. Right click->save as
	\begin{center}
e.g. overwrite the original source file from save as

e.g. the discover dev tools extension will be added at this point
	\end{center}
\end{document}
